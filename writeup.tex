\documentclass[12pt,a4paper]{article}

\usepackage[a4paper,text={16.5cm,25.2cm},centering]{geometry}
\usepackage{lmodern}
\usepackage{amssymb,amsmath}
\usepackage{bm}
\usepackage{graphicx}
\usepackage{microtype}
\usepackage{hyperref}
\setlength{\parindent}{0pt}
\setlength{\parskip}{1.2ex}

\hypersetup
       {   pdfauthor = { My Name },
           pdftitle={ Literate Programming Example },
           colorlinks=TRUE,
           linkcolor=black,
           citecolor=blue,
           urlcolor=blue
       }

\title{ Literate Programming Example }

\author{ My Name }

\date{ 6th Janurary 2020 }

\usepackage{upquote}
\usepackage{listings}
\usepackage{xcolor}
\lstset{
    basicstyle=\ttfamily\footnotesize,
    upquote=true,
    breaklines=true,
    breakindent=0pt,
    keepspaces=true,
    showspaces=false,
    columns=fullflexible,
    showtabs=false,
    showstringspaces=false,
    escapeinside={(*@}{@*)},
    extendedchars=true,
}
\newcommand{\HLJLt}[1]{#1}
\newcommand{\HLJLw}[1]{#1}
\newcommand{\HLJLe}[1]{#1}
\newcommand{\HLJLeB}[1]{#1}
\newcommand{\HLJLo}[1]{#1}
\newcommand{\HLJLk}[1]{\textcolor[RGB]{148,91,176}{\textbf{#1}}}
\newcommand{\HLJLkc}[1]{\textcolor[RGB]{59,151,46}{\textit{#1}}}
\newcommand{\HLJLkd}[1]{\textcolor[RGB]{214,102,97}{\textit{#1}}}
\newcommand{\HLJLkn}[1]{\textcolor[RGB]{148,91,176}{\textbf{#1}}}
\newcommand{\HLJLkp}[1]{\textcolor[RGB]{148,91,176}{\textbf{#1}}}
\newcommand{\HLJLkr}[1]{\textcolor[RGB]{148,91,176}{\textbf{#1}}}
\newcommand{\HLJLkt}[1]{\textcolor[RGB]{148,91,176}{\textbf{#1}}}
\newcommand{\HLJLn}[1]{#1}
\newcommand{\HLJLna}[1]{#1}
\newcommand{\HLJLnb}[1]{#1}
\newcommand{\HLJLnbp}[1]{#1}
\newcommand{\HLJLnc}[1]{#1}
\newcommand{\HLJLncB}[1]{#1}
\newcommand{\HLJLnd}[1]{\textcolor[RGB]{214,102,97}{#1}}
\newcommand{\HLJLne}[1]{#1}
\newcommand{\HLJLneB}[1]{#1}
\newcommand{\HLJLnf}[1]{\textcolor[RGB]{66,102,213}{#1}}
\newcommand{\HLJLnfm}[1]{\textcolor[RGB]{66,102,213}{#1}}
\newcommand{\HLJLnp}[1]{#1}
\newcommand{\HLJLnl}[1]{#1}
\newcommand{\HLJLnn}[1]{#1}
\newcommand{\HLJLno}[1]{#1}
\newcommand{\HLJLnt}[1]{#1}
\newcommand{\HLJLnv}[1]{#1}
\newcommand{\HLJLnvc}[1]{#1}
\newcommand{\HLJLnvg}[1]{#1}
\newcommand{\HLJLnvi}[1]{#1}
\newcommand{\HLJLnvm}[1]{#1}
\newcommand{\HLJLl}[1]{#1}
\newcommand{\HLJLld}[1]{\textcolor[RGB]{148,91,176}{\textit{#1}}}
\newcommand{\HLJLs}[1]{\textcolor[RGB]{201,61,57}{#1}}
\newcommand{\HLJLsa}[1]{\textcolor[RGB]{201,61,57}{#1}}
\newcommand{\HLJLsb}[1]{\textcolor[RGB]{201,61,57}{#1}}
\newcommand{\HLJLsc}[1]{\textcolor[RGB]{201,61,57}{#1}}
\newcommand{\HLJLsd}[1]{\textcolor[RGB]{201,61,57}{#1}}
\newcommand{\HLJLsdB}[1]{\textcolor[RGB]{201,61,57}{#1}}
\newcommand{\HLJLsdC}[1]{\textcolor[RGB]{201,61,57}{#1}}
\newcommand{\HLJLse}[1]{\textcolor[RGB]{59,151,46}{#1}}
\newcommand{\HLJLsh}[1]{\textcolor[RGB]{201,61,57}{#1}}
\newcommand{\HLJLsi}[1]{#1}
\newcommand{\HLJLso}[1]{\textcolor[RGB]{201,61,57}{#1}}
\newcommand{\HLJLsr}[1]{\textcolor[RGB]{201,61,57}{#1}}
\newcommand{\HLJLss}[1]{\textcolor[RGB]{201,61,57}{#1}}
\newcommand{\HLJLssB}[1]{\textcolor[RGB]{201,61,57}{#1}}
\newcommand{\HLJLnB}[1]{\textcolor[RGB]{59,151,46}{#1}}
\newcommand{\HLJLnbB}[1]{\textcolor[RGB]{59,151,46}{#1}}
\newcommand{\HLJLnfB}[1]{\textcolor[RGB]{59,151,46}{#1}}
\newcommand{\HLJLnh}[1]{\textcolor[RGB]{59,151,46}{#1}}
\newcommand{\HLJLni}[1]{\textcolor[RGB]{59,151,46}{#1}}
\newcommand{\HLJLnil}[1]{\textcolor[RGB]{59,151,46}{#1}}
\newcommand{\HLJLnoB}[1]{\textcolor[RGB]{59,151,46}{#1}}
\newcommand{\HLJLoB}[1]{\textcolor[RGB]{102,102,102}{\textbf{#1}}}
\newcommand{\HLJLow}[1]{\textcolor[RGB]{102,102,102}{\textbf{#1}}}
\newcommand{\HLJLp}[1]{#1}
\newcommand{\HLJLc}[1]{\textcolor[RGB]{153,153,119}{\textit{#1}}}
\newcommand{\HLJLch}[1]{\textcolor[RGB]{153,153,119}{\textit{#1}}}
\newcommand{\HLJLcm}[1]{\textcolor[RGB]{153,153,119}{\textit{#1}}}
\newcommand{\HLJLcp}[1]{\textcolor[RGB]{153,153,119}{\textit{#1}}}
\newcommand{\HLJLcpB}[1]{\textcolor[RGB]{153,153,119}{\textit{#1}}}
\newcommand{\HLJLcs}[1]{\textcolor[RGB]{153,153,119}{\textit{#1}}}
\newcommand{\HLJLcsB}[1]{\textcolor[RGB]{153,153,119}{\textit{#1}}}
\newcommand{\HLJLg}[1]{#1}
\newcommand{\HLJLgd}[1]{#1}
\newcommand{\HLJLge}[1]{#1}
\newcommand{\HLJLgeB}[1]{#1}
\newcommand{\HLJLgh}[1]{#1}
\newcommand{\HLJLgi}[1]{#1}
\newcommand{\HLJLgo}[1]{#1}
\newcommand{\HLJLgp}[1]{#1}
\newcommand{\HLJLgs}[1]{#1}
\newcommand{\HLJLgsB}[1]{#1}
\newcommand{\HLJLgt}[1]{#1}


\begin{document}

\maketitle

\section{Literate Programming}
This is a fancy way to make writeups that include text, math, plots, and code. It lets us avoid writing code in one file, evaluating it, and copying the output into LaTeX. The code that we are writing is evaluated in place.

\subsection{Text with Markdown}
We can use standard markdown like \textbf{bold} and \emph{italics}

\begin{itemize}
\item and


\item make


\item lists

\end{itemize}
\subsection{Code}
For example


\begin{lstlisting}
(*@\HLJLnf{print}@*)(*@\HLJLp{(}@*)(*@\HLJLs{"{}hello}@*) (*@\HLJLs{world"{}}@*)(*@\HLJLp{)}@*)
\end{lstlisting}

\begin{lstlisting}
hello world
\end{lstlisting}


Sometimes we'll want to share the code without showing the output Use the standard \texttt{;} to supress output


\begin{lstlisting}
(*@\HLJLni{1}@*)(*@\HLJLoB{+}@*)(*@\HLJLni{1}@*)
\end{lstlisting}

\begin{lstlisting}
2
\end{lstlisting}


\begin{lstlisting}
(*@\HLJLni{1}@*)(*@\HLJLoB{+}@*)(*@\HLJLni{1}@*)(*@\HLJLp{;}@*) (*@\HLJLcs{{\#};}@*) (*@\HLJLcs{supresses}@*) (*@\HLJLcs{output}@*)
\end{lstlisting}


\subsection{Plots}
Calling plot commands will output the plot directly in the file!


\begin{lstlisting}
(*@\HLJLk{using}@*) (*@\HLJLn{Plots}@*)

(*@\HLJLnf{plot}@*)(*@\HLJLp{(}@*)(*@\HLJLoB{-}@*)(*@\HLJLni{1}@*)(*@\HLJLoB{:}@*)(*@\HLJLnfB{0.01}@*)(*@\HLJLoB{:}@*)(*@\HLJLni{1}@*)(*@\HLJLp{,}@*)(*@\HLJLn{x}@*)(*@\HLJLoB{->}@*)(*@\HLJLn{x}@*)(*@\HLJLoB{{\textasciicircum}}@*)(*@\HLJLni{2}@*)(*@\HLJLp{)}@*)
\end{lstlisting}

\begin{figure}[!h]
\center
\includegraphics[width=\linewidth]{C:/Users/jiaqi.sun/Documents/Jackie/Personal/Master/UT Stats/Machine Learning/jl_OF7Ipu/writeup_4_1.pdf}
\caption{My Figure Caption}
\end{figure}

We can make the file in Julia by running

\subsection{LaTeX Math}
Get syntax highlighting with Atom \texttt{language-latex} package.

You can write inline math $x^2$ or block math like

\[
x^3
\]
\section{Compile the file into a pdf}
I reccomend evaluating the julia code using \texttt{Shift+Enter} as usual. But when you're ready to make your writeup you can run the following code!



\begin{lstlisting}
(*@\HLJLk{using}@*) (*@\HLJLn{Weave}@*)
(*@\HLJLnf{weave}@*)(*@\HLJLp{(}@*)(*@\HLJLs{"{}writeup.jmd"{}}@*)(*@\HLJLp{,}@*) (*@\HLJLn{doctype}@*) (*@\HLJLoB{=}@*) (*@\HLJLs{"{}md2pdf"{}}@*)(*@\HLJLp{)}@*)
\end{lstlisting}

and opening up \texttt{writeup.pdf}!



\end{document}
